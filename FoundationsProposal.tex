\documentclass[12pt]{article}
\usepackage[margin=1in]{geometry}
\usepackage{cite}
\usepackage{titlesec}

\begin{document}
\noindent
Tyler Shimko\\
006033765

\centerline{\textbf{Mechanism of X-silencing escape}}
\vspace{12pt}

X chromosome inactivation in mammals is controlled largely by the long noncoding RNA Xist. However, Xist-mediated silencing is incomplete and as much as 15\% of the "silenced" X chromosome may express genes to varying degrees \cite{Carrel:2005id}. This selective silencing along the X chromosome appears to play a role in the suppression of Turner Syndrome-like phenotypes in karyotypically normal females. For instance, the PAR1 locus on the X chromosome demonstrates pseudoautosomal behavior, with significant sequence homology with the Y chromosome and autosomal, rather than sex-linked, inheritance patterns for genes in the region \cite{Ciccodicola:2000up}. The presence of these genes on both the X and Y chromosomes offers support for the hypothesis that dose-dependent effects of the genes in the PAR1 and similar regions, may be responsible for the phenotypes associated with Turner Syndrome. Furthermore, this homologous region may explain why the XO karyotype leads to phenotypically-abnormal Turner Syndrome females while the XY karyotype leads to phenotypically-normal males, despite the presence of only one copy of the X chromosome in either individual.

Xist has been shown to work through a cis-regulatory mechanism, silencing genes in gene-rich regions of the chromosome first before moving outward along the chromosome toward gene-poor regions \cite{Simon:2013cz}. In this fashion, X silencing occurs most rapidly for areas of the chromosome in which dosage compensation may be most important. Recently, Jiang \textit{et al.} showed that the addition of an Xist allele to a single copy of chromosome 21 in trisomic induced pluripotent stem cell lines led to silencing of the transgenic copy, indicating that Xist-mediated silencing is rather flexible and not constrained by recognition of X-specific elements \cite{Jiang:2013ii}. However, the sparsity of genes measured for reduced expression in this study did not allow the authors to determine whether a similar number of genomic regions escape silencing on chromosome 21 as do on X. Consequently, the mechanism by which specific genomic regions escape Xist-mediated silencing remains largely enigmatic. Here, we propose to use induced pluripotent stem cells to investigate the mechanism by which regions of the X chromosome are protected against silencing.

\vspace{12pt}
\noindent
\textbf{Aim 1: Is the mechanism for avoiding X inactivation X-specific?}





\vspace{12pt}
\noindent
\textbf{Aim 2: Is the X inactivation center sufficient to induce proximal silencing in protected regions?}




\vspace{12pt}
\noindent
\textbf{Aim 3: Does a specific factor protect genomic sites from Xist-mediated inactivation?}


\pagebreak
\bibliography{XistBib}{}
\bibliographystyle{unsrt}
\end{document}
